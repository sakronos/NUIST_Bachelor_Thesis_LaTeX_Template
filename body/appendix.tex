\appendix

\section*{附录}
\phantomsection
\addcontentsline{toc}{section}{附录}

\subsection*{写在后面(第一版作者)}

时间过得真快,从五一动手,到码字码到这里差不多快三天了。这么短的时间,不管模板本身还是说明文档肯定还是不够完善的。但时间所迫,也必须到这了。

有人也许行会产生疑问,word不是用着挺好的吗,干嘛要学这个,干嘛要用这个写论文呢?其实要我回答呢,的确是这样的,随便用哪个排版软件用顺手了就好了,没人强迫你做什么,关键在于自己是怎么想。

去年笔者在写学年论文时,就“吃了亏”,先是用\LaTeX 写的,生成的pdf格式的文档,但是最后学院不认,说必须用word版的,无奈后来又用word重排了一遍。(所以这里插一句,如果真有哪位朋友想用这个模板,请“严重”地考虑这个“严重”的后果,弄不好到最后,只能用它排个打印版玩玩,电子版还得word 去。)

而对笔者来说,无所谓,不是天空经常会飘来五个字儿,叫“这都不是事儿”嘛,人生本就是向死之生,要是总走直线,太快到终点了怎么办?所以人生的要义就在于走“弯路”,走得越“弯”走得越长嘛。

\subsection*{第二版修订说明}

\subsubsection*{第二版!新鲜出炉!}

两年以后\footnote{第二版修订于2017年3月13日,\url{https://github.com/LirenW/NUIST_thesis_template_V2.0}},这个模板又被重新更新了一次(原作者应该并没有更新过(因为并不能联系到原作者(sigh。因为每年的格式都会进行一些修改,所以按照现在的格式改了一下模板,特别是字号和字体,并且针对一些问题进行了修改(以下如果感觉麻烦可以略过233),不过要注意的是,NUIST一向不欢迎PDF格式的论文提交,因此此模板,正如原作者说的那样,需要慎用、慎用和慎用。\par
对于无法复制PDF的问题,由于CTeX的设置问题解决方案比较复杂,本模板采用修改字体为Adobe Song Std 的方法,不过如果要完整解决此问题请参考\url{https://www.zhihu.com/question/32207411}这个回答,不过低版本的CTeX+WinEdit套装中CTeX版本过低无法使用,可以考虑升级全部宏包(此方法可能会导致WinEdit宏包冲突,慎用)也可以等新版本的套装(听说快出了)。\par
关于行距的问题,虽然word和LaTeX的行距计算方法相同(行距:一行文字的基线(Base Line)到下一行文字的基线的距离,详见\cite{x4}),但是修改出的文章行距感觉比word略宽,不知道为什么,期待后人能解决此问题!\par

\subsubsection*{修订者的话}

说完了专业问题,聊点其他的话题好了。笔者接触LaTeX也蛮久了,从数学建模就用自己修改的模板进行论文写作,到写毕业论文时还是用LaTeX,感觉长文章基本脱离Word了,不是不会用(不自夸地说,论Word排版本人也完全可以完成长文章的各种排版工作),而是感觉Word排出来的东西一点也不美。\par
Knuth感觉自己写的东西被编辑排成了渣,于是很不开心地花时间做了个排版系统;乔布斯觉得手机太丑,于是自己做了个iPhone。我也有这种感觉,且不说Word那蛋疼的贪心断行算法(最常见的例子的是加上了数学公式和英文字符后完全不对齐的右边界)、令人抓狂的图片摆放,就拿最简单的来说,一个写作软件,为什么要让用户找不到如何更新引用!我知道那复杂的域代码和目录生成,然而一个一个设置它们的格式实在令人发指,并且一个不小心,版面就跑到十万八千里以外。不过什么是美呢?想来对我来说的话就是“Simple is the best”,能让电脑自动计算的事情完全不应该由手工来做,能动脑解决的就绝不动手。\par
世界是因为懒人才变得舒适,但“懒”往往需要的是Critical Thinking和Curiosity,而对我来说,对美这一形而上的终极目标的追求促使我探索这个世界,而对这个世界无穷无尽的美好的好奇让我在探索的过程中不太无聊。\par
引用百度百科(好吧我最唾弃百度的各种玩意了)TeX的词条的一句话吧:

\begin{quote}
  TeX是一种乐趣: 使用TeX不仅仅是一种工作手段,也是一种乐趣。它有挑战,也有荣誉。很多人在熟悉了TeX之后都开始把使用TeX作为一种爱好,而不是一件枯燥无味的劳动。
\end{quote}

我使用TeX就是因为它简洁明快,让我专注于内容而不需要纠结于无聊的排版疏忽,随意调节结构而不用担心随之而来的格式更新,总而言之就是这个样子\footnote{面白い}。\par
\subsection*{2021.6版修订说明}

\subsubsection*{南信大与PDF格式论文}

首先,笔者要和前面两位唱个反调,是时候打破“南京信息工程大学不欢迎PDF格式论文”这个传言了。南信大论文系统提交文件处写明“格式建议:word,pdf”,在笔者撰写论文前也确认过可以提交PDF格式的论文,最重要的是,笔者自己提交的就是PDF格式的论文。南信大并不是不允许PDF格式的论文。当然,笔者能够全程使用\LaTeX 撰写论文离不开笔者的毕业设计指导老师的支持,因为今年(2021年)的《关于毕业论文(设计)材料归档工作的通知》里还是写了“上传论文须WORD格式,PDF格式的论文和设计实现的系统/软件作为附件打包上传至系统。”,不过指导老师允许笔者最后的归档文件无须提交Word文档。

如果您希望使用\LaTeX 撰写论文,建议您向论文指导老师确认对\LaTeX 的态度。下面引用《关于毕业论文(设计)材料归档工作的通知》的部分段落:

\begin{quote}
    一、需归档的材料

    1、任务书;2、开题报告;3、中期检查表;4、外文翻译;5、毕业论文定稿(word和PDF格式);6、指导教师审阅意见表;7、系统或其他附件

    二、归档要求

    所有材料的电子版均需保存或上传到“毕业设计(论文)智能管理系统”(下称“系统”)
    注:1、上传论文须WORD格式,PDF格式的论文和设计实现的系统/软件作为附件打包上传至系统。如果是软件,还需要写一份软件说明书,说明具体的操作步骤;如果是硬件,建议将硬件保留下来,将硬件演示过程拍一段视频,上传至系统。
\end{quote}

\subsubsection*{更新说明}

本次修订\footnote{网址:\url{https://sakronos.github.io/NUIST_Bachelor_Thesis_LaTeX_Template/}}根据南信大2021年本科毕业论文格式要求对原有模板进行修订,参考了《南京信息工程大学LaTeX毕业论文模板V3.1》\cite{geiNanJingXinXiGongChengDaXueLaTeXBiYeLunWenMoBanV31GengXinWuXuYiLaiCTeXRuanJian2021}。关于页码、声明页、按章编号等《南京信息工程大学本科生毕业论文(设计)撰写排版规范》没有提及的额外排版要求则是根据笔者导师要求设定的,如果与您所在学院老师要求发生冲突,请报告。

由于时间较长,笔者无法一一列出本次修改的具体内容,这里根据记忆尽量列出修订内容:
\begin{enumerate}[1、]
    \item 调整了几处字体大小
    \item 将图片、表格、公式设置为按章编号
    \item 添加了声明页
    \item 设置了页码
    \item 使用GB/T 7714—2015 BibTeX Style排版参考文献
    \item 限定模板使用的字体为SimSun、SimHei、SimKai和Times New Roman
    \item 替换已弃用的宏包和命令
    \item 更新\verb|\thanking|命令,添加\verb|\forthsection|命令
    \item 将\verb|\linespread|设置为1.335,以得到更接近MS Word下多倍行距1.25的效果
    \item 图片编号与图片标题间的分隔符设置为空格
    \item 更新模板介绍(本PDF文档)
\end{enumerate}

笔者在使用本模板的过程中没有遇到“文字无法复制的问题”,如果有同学遇到该问题请报告。

\subsubsection*{闲话}

虽然很讨厌写字,但是笔者还是写一点闲话吧。

不像该模板的创建者和第一位修订者,笔者之前并没有使用\LaTeX 的经验。笔者是在写论文的过程中不断摸索\LaTeX 的使用方法,对\LaTeX 的了解很少,因此笔者怀着诚惶诚恐的心情修订这份模板。各位如果能指出模板和本文中的错误,笔者会非常开心的。笔者也期待各位加入本模板的修订工作,笔者的文字功力太差,难免写出晦涩难懂的语句,需要各位帮助补充/润色模板文档。

下面是吐槽,Windows 系统下的TeX Live Manager这个图形化工具做的很是不好,更新Packages时不能最小化。刚刚笔者用Windows的显示桌面强行最小化这个工具后,无法还原到桌面了!!!笔者现在不知道更新的进度,只能等它在后台更新完……以后还是老老实实地用命令行更新了。(现在发现能用任务管理器强行最大化TeX Live Manager)

\subsubsection*{致谢}

本次修订首先要感谢本模板的制作者和2.0版修订者,如果没有这两位的工作,我不会鼓起勇气使用\LaTeX 撰写毕业论文,本次修订也是在这两位的工作基础上进行的。

然后,感谢我的毕业论文指导老师,感谢老师指出论文排版不美观的地方,帮助我改进该模板。

最后,感谢《南京信息工程大学LaTeX毕业论文模板V3.1》的制作者。虽然本次修订工作与这位的算是各自进行,但是您的工作给了我不少启发,也激励我在提交论文后继续完善本模板。您的CLS文件层级分明,值得学习。遗憾的是您留下的邮箱地址不存在,无法与您取得联系。
\subsection{2022版修订说明}

\subsubsection{更新内容}

\begin{enumerate}[1、]
    \item 封面信息允许换行,以免遇到如“计算机学院、软件学院、网络空间安全学院”这样实在写不下的学院名称
    \item 解决参考文献作者全部大写的问题
\end{enumerate}

