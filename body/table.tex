\section{排版表格}

\LaTeX 中生成简单的表格还是比较方便的,可以用tabular 环境来实现。模板引入了 booktabs 包实现三线表样式,下面就来做一个论文中经常用到的三线表,如表~\ref{table_1}~。

\begin{table}[htbp!]
    \centering
    \caption{本模板中部分使用的宏包及功能}
    \label{table_1}
    \begin{tabular}{ccccccccc}
        \toprule
        宏包名称 & amsmath  & caption  & geometry & ulem   & xcolor & setspace & hyperref \\
        \midrule
        作用     & 数学公式 & 定制标题 & 页面设置 & 下划线 & 颜色   & 行距     & 超链接   \\
        -        & -        & -        & -        & -      & -      & -        & -        \\
        \bottomrule
    \end{tabular}
\end{table}

代码与普通表格类似,将 \verb|\hline| 分别换成 \verb|\toprule| \verb|\midrule| \verb|\bottomrule| 即可,其实现代码如下:

{
\color{green!50!black}
\begin{lstlisting}[breaklines=true,]
\begin{table}[htbp!]
    \centering
    \caption{本模板中部分使用的宏包及功能}
    \label{table_1}
    \begin{tabular}{ccccccccc}
        \toprule
        宏包名称 & amsmath  & caption  & geometry & ulem   & xcolor & setspace & hyperref \\
        \midrule
        作用     & 数学公式 & 定制标题 & 页面设置 & 下划线 & 颜色   & 行距     & 超链接   \\
        -        & -        & -        & -        & -      & -      & -        & -        \\
        \bottomrule
    \end{tabular}
\end{table}
\end{lstlisting}
}

如果表格比较长,那就要用到跨页表格排版宏包longtable了(模板中已引入该宏包)。基本的表格排版情况就介绍这么多,大家感兴趣自己慢慢去探索吧。

对于复杂表格,可以使用在线工具 Tables Generator \url{https://www.tablesgenerator.com/}
